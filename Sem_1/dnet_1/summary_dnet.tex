\documentclass[a4paper,12pt]{article}
\usepackage{array}
\usepackage{listings}
\usepackage{graphicx}
\graphicspath{ {./images/} }

\setlength{\tabcolsep}{18pt}
\renewcommand{\arraystretch}{1.2}


\begin{document}
\title{Zusammenfassung Datennetze 1}
\author{Mathis Hermann}
\date{\today}
\maketitle
Diese Zusammenfassung ist anhand der Lernziele aufgebaut. Dementsprechend sind nicht ganz alle Themen der Vorlesung enthalten. It's been a f*cking pain...

\section{Übersicht über Datennetze}



\paragraph{Client-Server Paradigma} -- Meiste Kommunikation; Daten auf Server; User können Daten auf Server bearbeiten

\paragraph{Peer-to-Peer Paradigma} -- Rechner kann Server und Client sein; Einfach aufzusetzen, weniger Komplexität, weniger Kosten; Keine zentralisierte Administration, nicht so sicher, nicht skalierbar

\paragraph{LAN und WAN} -- Es gibt verschiedene Typen von Netzen:\\
Kriterien:
\begin{itemize}
\item Kanalzugang
\begin{itemize}
\item Multiaccess Netze
\item Punkt-zu-Punkt Netze
\end{itemize}
\item Ausdehnung
\begin{itemize}
\item LAN (sind immer Multiaccess Netze)
\item WAN (traditionell: Punkt-zu-Punkt Netze; neu auch Multiaccess Netze)
\end{itemize}
\end{itemize}

\paragraph{Begriffe}
\begin{center}
\begin{tabular}{ | m{2.5cm} | m{9cm} | } 
Begriff & Erklärung\\ 
\hline
Client & Wird ein- und ausgeschaltet; wechselt IP; stellt Anfrage an Server\\
Server & Ein Programm, das immer läuft; fixe IP; meistens DNS- Eintrag; läuft ununterbrochen und wartet auf Anfragen\\
LAN & Local-Area Network; Enthält  Endgeräte; Medium-Access Control; \emph{multi-access} Netz \\
WAN &Wide-Area Network; Keine Endgeräte; \emph{point-to-point} Leitungen\\
Physikalische Darstellung & Definiert die physikalischen Verbindungen und Konfigurationen; Zeigt welches Gerät sich wo befindet und mit welchem verbunden ist\\
Logische Darstellung & Darstellung Ebene 3; Netze werden gezeichnet\\
Switch & Layer 2 Kommunikation; leitet Rahmen innerhalb eines Netzwerkes weiter\\
Router & Layer 3; Verbindet IP-Netze; leitet Pakete von einem Netz in ein anderes\\
Konvergierte Netze & Ein Netz für alle Dienste; single point of failure im Netz durch Komplexität; billiger im Unterhalt\\
Downstream & (Stärke der) Leitung von ISP zu Konsument\\
Upstream & (Stärke der) Leitung von Konsument zu ISP\\
QoS & Dienstgüte\\
Leitungs-vermittlung & Endgerät baut Leitung zu Ziel auf; feste Bandbreite wird reserviert in festem zeitlichen Rahmen; starke QoS; geringer Durchsatz (stark begrenzt in Bandbreite); sicher\\
Paket-vermittlung & Nachricht wird segmentiert; Pakete werden anhand der Informationen (Absender, Empfänger) verteilt; gute Ausnützung der Bandbreite; nicht so sicher; viel Overhead bei Paketen - nicht so effizient\\
\end{tabular}
\end{center}

\paragraph{Netzwerkelemente und deren Funktion} -- Netzwerkelemente sind \emph{Endgeräte} (Computer, Laptop, Drucker, Tablet); \emph{Zwischengeschaltete Geräte} (Router, Switch) ; \emph{Anschlussleitung} (Wireless Media, LAN Media, WAN Media)

\paragraph{Dedizierte und Konvergierte Netze} -- verschiedene Dienste über das Internet\\
\emph{Dedizierte Netze} -- Spezifische Netze für entsprechende Anwendungen; \emph{Konvergente Nezte} -- Ein Netz für alle Dienste;

\paragraph{Logische und Physikalische Darstellung von Netzwerken} -- \emph{Logische Sicht} Beschreibt welche IP-Netze (Benutzergruppen) es gibt; \emph{Physikalische Sicht} Beschreibt wo welches Netzelement befindet

\paragraph{Grundanforderungen des Internets}
Kriterien für zuverlässige Netze:

\begin{itemize}
\item Fehlertoleranz (z.B. bei Leitungunterbruch)
\item Skalierbarkeit
\item Dienstgüte
\item Sicherheit (e.g. Abhören oder Manipulation der Daten)
\end{itemize}

\paragraph{Anforderung verschiedener Anwendungen an die verschiedenen Parameter der Dienstgüte}

\begin{center}
\begin{tabular}{ | m{1.2cm} |m{1.2cm}|m{2cm}|m{1cm}|m{2cm}| } 
An-wendung & Durch-satz & Ver-zögerung & Jitter & Paket-vermittlung\\ 
\hline
Web & Hoch & Gering & Gering & Hoch\\
E-Mail & Gering & Gering & Gering & Gering\\
Sprache & Gering & Hoch & Hoch & Gering (VoIP)\\
Video & Mittel & Hoch & Hoch & Gering\\
\end{tabular}
\end{center}

\newpage
\paragraph{Unterschied: Leitungsvermittlung und Paketvermittlung} \emph{Leitungsvermittlung} -- Viele mögliche Pfade; ein Pfad wird gewählt pro Call; Wenn ein Call etabliert ist, geht alle Kommunikation über diesen Pfad; Eine Leitung ist bestimmt für die gesamte Dauer des Calls\\
\emph{Paketvermittlung} -- Viele verschiedene Pfade können verwendet werden, um individuelle Pakete zum Ziel zu routen; kein fixer Pfad; Pakete werden entsprechend des besten Pfades zur Zeit geroutet \\

\newpage
\section{Konfiguration von Netzen}
\paragraph{Zugänge zum Betriebssystem IOS von Cisco} -- Hauptsächlich über drei HW-Schnittstellen kann ein Router konfiguriert werden: \emph{1)} Console Port, blau: Management Port für lokale Konfiguration (Lokale Konsole, Serial), \emph{2)} Auxiliary Port, schwarz: Management Port für entfernte Konfiguration ("Remote out-of-band", wird kaum mehr genutzt), \emph{3)} LAN Anschlüsse, gelb: Remote Management ("inband management") mit SSH, Telnet.



\paragraph{Kommandos IOS} -- Drei verschiedene Arten von Kommandos in IOS: \emph{1)} Betriebskommandos ("ping"; Speichern von Konfiguration etc), \emph{2)} Konfigurationskommandos (zur Konfiguration eines Interfaces, Passwort etc.), \emph{3)} Statusabfragen.

\paragraph{Kommandostruktur von IOS} -- Es gibt vier verschiedene Modi im IOS: \emph{1)} User Mode, \emph{2)} Privileged Mode, \emph{3)}Global Config Mode, \emph{4)} Config Mode (und Sub-Menus).


\paragraph{Wie wird ein Router oder Switch out-of-the-box konfiguriert?}
\begin{lstlisting}
Router>enable	// exec zu privileged mode
Router#configure terminal // in Konfigurationsmodus
Router(config)#interface FastEthernet 0/0
// IF-Konfigurationsmodus - Interfaces konfigurieren

Router(config-if)# // Ebene zurueck
Router(config)#	// Ebene zurueck
Router#disable	// zurueck in den user mode
Router>

// Hilfe im IOS

Router# cl? //liste von commands, die mit "cl" starten
clear clock

Router#clock set ? // next possible arguments
  hh:mm:ss	Current Time
  
Router#clock set 19:50:00 ? // mehrere Argumente
  <1-31>	Day of the month
  MONTH		Month of the year

Router#clock set 19:50:00 25 June 2022

// Konfiguration auslesen
Router#show running-config
	// laufende Konfiguration aus RAM

Router#show startup-config
	// Abgespeicherte Konfiguration aus NVRAM
	
Router#show flash // Inhalt von Flashspeicher
Router#show version // Informationen zum aktuellen IOS
Router#show ip interface brief // Zustand aller IFs
\end{lstlisting}

\paragraph{Grundkonfiguration für Router und Switch} Eine Grundkonfiguration beinhaltet:
\begin{itemize}
\item Hostname
\item Einschränkung des Zugangs zu Netzelementen mit Passwörtern
	\begin{itemize}
	\item Lokales Einloggen (Console)
	\item Entferntes Einloggen (Telnet, SSH)
	\item Übergang user mode zu privileged mode
	\end{itemize}
\item Rechtlicher Hinweis mit einem Banner
\end{itemize}

\begin{lstlisting}
Router>enable
Router#configure terminal

// Hostname
Router(config)#hostname NAME
NAME(config)#

// Loeschen eines Konfigurationsbefehls
NAME(config)#no hostname NAME
Router(config)#

// PW fuer lokales Einloggen
Router(config)#line console 0
Router(config-line)#password PASSWORD
Router(config-line)#login

// PW fuer entferntes Einloggen (telnet)
Router(config)#line vty 0 4 // sesions 0-4
Router(config-line)#password PASSWORD
Router(config-line)#login
Router(config-line)# transport input telnet

// PW fuer privileged mode ohne Verschluesselung
Router(config)#enable password PASSWORD
// PW fuer  privileged mode mit Verschluesselung
Router(config)#enable secret PASSWORD

// Verschluesselung aller PW in Konfiguration
Router(config)#service password-encryption


// Rechtlicher Hinweis
Router(config)#banner motd "
---------------------------------
Authorized Access Only!
---------------------------------
"

// Speichern von Konfiguration in NVRAM
Router#copy running-config startup-config

// Speichern auf TFTP-Server
Router#copy running-config tftp

// Speichern in Datei
Router#show running-config // copy & paste in externe Datei

// Loeschen von lokal gespeicherten Konfigurationen
// Router
Router#erase startup-config
// Switch
S1#erase startup-config
S1#delete flash:vlan.dat

\end{lstlisting}

\paragraph{IP-Adresse an Interface konfigurieren}
\begin{lstlisting}
// Router
    // interface konfigurieren
Router(config)#interface Fastethernet 0/0
Router(config-if)#ip address IF-ADRESSE NETZMASKE
Router(config-f)#no shutdown // IF einschalten

// Switch
S1(config)#interface vlan 1
S1(config-if)#ip address IF-ADRESSE NETZMASKE
S1(config-if)#no shutdown
\end{lstlisting}


\paragraph{Zustand von Leitungen abfragen und Erreichbarkeit von Netzelementen prüfen}
Wenn die IP-Adresse bekannt ist, kann mit \verb+ping+ geprüft werden, ob ein Netzelement erreichbar ist.


\newpage
\section{Netzwerkprotokolle und Datenkapselung}

\paragraph{Protokoll}
Bei Rechnernetzen gibt es Regeln, wie Nachrichten auszutauschen sind:
\begin{itemize}
\item \emph{Kodierung der Daten} (wie wird ein Zeichen kodiert)
\item \emph{Format für die Kapselung der Nachricht} (welche Informationen werden wie ausgetauscht)
\item \emph{Maximal erlaubte Grösse}
\item \emph{Zeitliche Abfolge} (Wie erfolgt der Zugang zum Übertragungskanal? Wie schnell darf ein Sender senden?)
\item \emph{Optionen}, auf die sich beide Seiten einigen können
\item \emph{Fehlerbehandlung}
\end{itemize}

Ein Protokoll ist ein \emph{Satz von Regeln}, der die Form der Kommunikation hinreichend genau bestimmt, so dass Rechner verschiedener Hersteller effizient miteinander Daten austauschen können.

\paragraph{Protokolle im TCP/IP-Modell}
TCP/IP ist einer der meist angewendeten Protokoll-Stapel. Dieser beinhaltet verschiedene Protokolle und deren Integration.

\begin{center}
\begin{tabular}{ | m{2cm} |m{7cm}| } 
Schicht & Protokolle\\ 
\hline
Application Layer & DNS (Name System), DHCP (Host Config), SMTP (Email), FTP (File Transfer), HTTP (Web) \\
Transport Layer & UDP, TCP \\
Internet Layer & IP, ICMP (IP Support), OSFP (Routing Protocols) \\
Network Access Layer & ARP, PPP, Ethernet\\
\end{tabular}
\end{center}


\paragraph{Standardisierungsbehörden}
\begin{itemize}
\item The Internet Architecture Board (IAB)
\item The Internet Engineering Task Force (IETF) -- Standardisiert viele Protokolle
\item Internet Assigned Numbers Authority (IANA) -- vergibt IP Adressen
\item The Institute of Electrical and Electronics Engineers (IEEE) -- kontrolliert Protokolle in der unteren Schicht
\item The International Standardization Organization (ISO)
\end{itemize}


\paragraph{Aufgaben der Schichten im OSI-Modell}


\begin{center}
\begin{tabular}{|c| m{1.9cm} |m{7.5cm}| } 
\#&Schicht & Aufgabenbereich\\ 
\hline
7 & Anwendung & Stellt die Dienste und Funktionalitäten für den Anwender bereit\\
6 & Darstellung & Verwaltet die Darstellungsinformation des Dateninhalts (Komprimierung, Verschlüsselung)\\
5 & Sitzung & Verwaltet die verschiedenen Sitzungen zwischen den Endpunkten\\
4 &Transport & Stellt der Anwendung zwei Dienste für die Übertragung der Daten von der Quelle zum Ziel bereit (Endgerät zu Endgerät)\\
3 & Netzwerk & Wegleitung vom Netz der Quelle zum Zielnetz \\
2 & Sicherung & Legt fest, wie die Daten innerhalb eines Netzes ausgetauscht werden\\
1 & Physik & Legt die elektrischen Signalformen für die Übertragung innerhalb eines Netzes fest\\
\end{tabular}
\end{center}

\newpage
\paragraph{Mechanismen beim Versenden von Daten}
Grosse Datenblöcke verschiedener Teilnehmer werden beim Sender segmentiert und "gemultiplext" über eine Leitung übertragen. Jede Leitung legt eine maximale Grösse eines einzelnen Rahmens (MTU = maximal transmission unit) fest. Datenkapselung: Protocol Data Unit (PDU) = [Header, Payload]



\emph{Kapselung verschiedener Protokolle beim Sender} -- Die Sendeseite fügt einen Header hinzu und übergibt die PDU der darunter liegenden Schicht.

\emph{Entkapselung beim Empfänger} -- Die Empfangsseite wertet den Header aus, entfernt ihn wieder und übergibt die Payload der darüberliegenden Schicht.

\paragraph{Adressen in den verschiedenen Schichten}

\paragraph{Abläufe beim Senden einer Dateneinheit}
\emph{Ziel im gleichen Netz} -- Schicht 2 MAC-Adressen (Destination, Source), Schicht 3 IP (Source, Destination), Data; Header bleiben gleich

\emph{Ziel in anderem Netz} -- Schicht 2 MAC-Adressen (Destination, Source), Schicht 3 IP (Source, Destination), Data; Schicht 3 Header bleibt gleich, Schicht 2 Header wird ausgetauscht.

\begin{itemize}
\item Die Schicht 3 geht von Ende zu Ende
\item Die Schicht 2 wird beim Übergang von einem Netz in ein anderes neu gebildet
\end{itemize}

\begin{center}
\includegraphics[width=12cm]{img/02_transport_data.png}
\end{center}

\newpage
\section{Schichten 1 und 2: Network Access}

\paragraph{Aufgaben Schicht 1}
Die physikalische Schicht (Layer 1) bietet der Sicherungsschicht (Layer 2) den Dienst an, Bits über einen Link zu übertragen. Ethernet umfasst Layer 1 und Layer 2.

\paragraph{Wichtigste Physikalische Schnittstellen an Netzelementen}

Folgende physikalische Komponenten werden standardisiert:
\begin{itemize}
\item Netzwerkanschlüsse (Network Interface Card -- NIC)
\item Übertragungsmedien (Leitungen)
\item Steckverbinder
\item Elektrische Signale
\item Kodierung
\end{itemize}


\paragraph{Drei wichtigste Übertragungsmethoden und deren Eigenschaften}

\begin{itemize}
\item \emph{Synchrone Übertragung} -- Das Taktsignal wird mit dem Datensignal mitübertragen
\item \emph{Asynchrone Übertragung} -- Nur das Datensignal wird übertragen. Der Empfänger benötigt eine \emph{Taktrückgewinnung}
\end{itemize}

Im Allgemeinen wird asynchron übertragen. Der Empfänger muss das Taktsignal aus dem Empfangssignal zurückgewinnen. Das Ziel ist es, \emph{1)} so viele Daten pro Zeiteinheit wie möglich und \emph{2)} so weit wie möglich zu übertragen. Die Dämpfung elektrischer Signale in einem Leiter nimmt i.A. mit der Frequenz zu.

Das Empfangssignal ist ein analoges Signal. Der Empfänger entscheidet um welches Symbol es sich handelt. Dabei können Fehler auftreten.

\paragraph{Wichtigste Kabel und wie sie eingesteckt werden}
\begin{itemize}
\item \emph{Unshielded Twisted-Pair Cable} -- Verschiedene Kabel mit verdrillten Kupferadern
\item \emph{Shielded Twisted-Pair Cable} -- Verschiedene Kabel mit verdrillten Kupferadern; abgeschirmt
\item \emph{Coaxial cable}
\end{itemize}



\begin{center}
\begin{tabular}{|m{1cm}| m{3cm} |m{2cm}|m{3cm}|} 
Medium& Eigenschaften & Vorteile & Nachteile\\ 
\hline
Kupfer & Twisted / Untwisted & Billiges Material & Dämpfung hoch bei hohen Frequenzen\\
Glas & Multimode (mehrere Pfade für das Licht), Singlemode (Pfad in eine Richtung) & Störungs-unabhängig, dämpfungs-arm & Teure Hardware\\
Luft & Signal wird auf Träger moduliert & Mobilität & Hohe Dämpfung bei hohen Frequenzen, alle können Signale empfangen und mithören, Interferenzen mit anderen Sendern, beschränkte Bandbreite wird geteilt\\
\end{tabular}
\end{center}


\paragraph{Leitungscode}

Der Sender wandelt das binäre Signal zuerst in einen dem Kanal angepassten Leitungscode um. Der Leitungscode legt fest, wie das Signal übertragen wird. Dabei ist es wichtig, das Signal dem Übertragungsmediom entsprechend anzupassen, um eine optimale Übertragung zu erreichen.

\paragraph{Aufgaben der Sicherungsschicht (data link layer)}

\begin{itemize}
\item Rahmenbildung um Pakete zu begrenzen
\item Regelung des Zugriffs auf den Übertragungskanal
\item Fehlererkennung
\item Optional (von Ethernet nicht unterstützt)
	\begin{itemize}
	\item Aufbau, Halten und Auflösen einer Verbindung
	\item Fehlerkorrektur durch Übertragungswiederholung
	\item Flusssteuerung
	\end{itemize}
\end{itemize}

Dafür werden Protokolle der Sicherungsschicht verwendet: \emph{LAN} -- Ethernet2, Ethernet IEE 802.3, TokenBus IEEE 802.4, Token Ring IEEE 802.5, WLAN IEEE 802.11; \emph{WAN} -- HDLC (High-Level Data Link Control), PPP, FrameRelay

\paragraph{Allgemeine Struktur eines Schicht-2-Rahmens}
Die Schicht 2 legt die Felder des Rahmens und die maximale Länge der Daten (Maximum Transmission Unit, MTU) in einem Rahmen fest.

\begin{center}
\includegraphics[width=12cm]{img/04_L2_frame.png}
\end{center}

\paragraph{Verschiedene logische Topologien in LAN}

\begin{itemize}
\item Stern
\item Erweiterte Stern
\item Bus
\item Ring
\end{itemize}

\begin{center}
\includegraphics[width=12cm]{img/04_lan_topology.png}
\end{center}

\paragraph{LAN} Ursprünglich Shared Medium. Viele Stationen benützen denselben Kommunikationskanal. Weiterentwicklung: \emph{switched networks}

\emph{Half-Duplex} -- Ein Endgerät kann zu einem Zeitpunkt entweder senden oder empfangen (Ethernet mit einem Hub; Wireless Access Point)

\emph{Half-Duplex} -- Beide Enden können gleichzeitig senden und empfangen (Ethernet mit Switches).

\paragraph{Maximum Transmission Unit}
\begin{itemize}
\item Jedes L2-Protokolldefiniert eine MTU
\item Ein Rahmen darf nicht grösser sein als die MTU; ggf. muss das Schicht 3 Protokoll einen Rahmen fragmentieren, wenn er für einen Link zu gross ist
\item Typische MTU: 1500 Bytes
\item Grund: Ein Benutzer darf einen Kanal nicht zu lange belegen damit andere auch senden / empfangen können
\end{itemize}

\paragraph{Verschiedene logische Topologien in WAN}
\begin{itemize}
\item Punkt-zu-Punkt -- dedizierte Leitungen
\item Point-to-Multipoint -- Hub and Spoke, partial mesh; Hub Standort mit mehreren Standorten verbunden
\item Full Mesh -- Alle Geräte können miteinander kommunizieren
\item Ring -- Weniger anfällig auf Fehler wegen Verbindung im Kreis; kein single-point-of-failure
\item Stern -- nur zentraler Hub kann single-point-of-failure werden
\end{itemize}


\paragraph{Kategorien des Kanalzugriffsverfahren}
\emph{Controlled Access} -- Jede Station hat eine Zeit, die für sie zum Senden reserviert ist. In dieser Zeit darf nur sie senden (Token Ring, FDDI).

\emph{Contention-based Access} -- Als Wettbewerb: Der schnellere darf senden. Es ist ein Verfahren definiert, wie vorzugehen ist, wenn zwei Stationen gleichzeitig senden.

\newpage
\section{Ethernet}

\paragraph{OSI-Schichten von Ethernet abgedeckt} -- Ethernet deckt die Schichten 1 (Physikalisch) und 2 (Data Link, Sicherung) ab.

\paragraph{Aufgaben der MAC-Schicht}
\begin{itemize}
\item Kapselung der Daten
	\begin{itemize}
	\item Markierung des Beginns eines Rahmens
	\item Adressierung
	\item Fehler Detektion
	\end{itemize}
\item Kontrolle des Kanalzugangs
	\begin{itemize}
	\item Platzierung der Rahmen auf den Kanal
	\item Fehlerbehandlung bei Kollisionen
	\end{itemize}
\end{itemize}


\paragraph{Physikalische Unterlagen von Ethernet}
\begin{itemize}
\item Coax N-Style -- 500m
\item Coax BNC -- 185m
\item UTP RJ45 -- 100m
\item STP mini-DB-9 -- 25m
\item MM Fiber SC -- 220-550m
\item MM Fiber SC -- 550-5000m
\end{itemize}

\paragraph{Topologie vom ursprünglichen Ethernet}
\begin{itemize}
\item Offen
\item Einfachheit, einfacher Unterhalt, Zuverlässigkeit
\item Neue Technologien integrieren, ohne Alte ersetzen zu müssen.
\item Günstig in Installation und Aufrüstung
\end{itemize}

Mit Koaxkabel ("Bus") wurde das Ethernet ursprünglich entwickelt.

\paragraph{Kollisionsdomänen} -- Kollisionsdomänen werden unterteilt nach Geräten, die gleichzeitig senden / empfangen können. Wenn zwei Rechner in der gleichen Kollisionsdomäne sind, kann je nur einer der beiden gleichzeitig senden / empfangen.

Kollisionsdomänen können mit Hub erweitert (mehr Geräte einschliessen) und mit Switch oder Bridge in mehrere aufgetrennt werden.

\paragraph{Ethernet-Adressen} -- Allgemein werden entsprechend dem Adressaten drei Arten von Paketen unterschieden
\begin{itemize}
\item \emph{Unicast} -- Ziel ist ein Rechner
	\begin{itemize}
	\item Unicast-Adressen -- LSB des ersten Byte ist Null
	\end{itemize}
\item \emph{Multicast} -- Ziel ist eine Gruppe von Rechnern oder NIC
	\begin{itemize}
	\item Die IP-Adressen von 224.0.0.0 - 231.255.255.255 repräsentieren Multicast-Gruppen
	\item Wenn sich eine Anwendung an einer Multicast-Gruppe anmeldet, wird dem Rechner eine IP-Adresse vom Multicast-Bereich zugeordnet
	\item Zuordnung geschieht auf Schicht 3. Schicht 2 wird nachgezogen (Für IP-MC muss auch MAC-Adresse gebildet werden)
	\end{itemize}
\item \emph{Broadcast} -- Ziel sind alle NIC in einem Netz
	\begin{itemize}
	\item Ziel-Adresse lautet FF:FF:FF:FF:FF:FF
	\end{itemize}
\end{itemize}

\paragraph{Aufbau MAC-Adresse} -- MAC-Adressen bestehen aus zwei Teilen (je 3x 2 Oktet). Somit werden die ersten 6 Oktet der dem Hersteller (Organisationally Unique Identifier, OUI) vergeben. Der zweite Teil kann der Hersteller frei vergeben (Vendor Assigned; NIC, Interfaces)

Es gibt $2^{48}$ mögliche MAC-Adressen. Eine Organisation kann also $2^{24}$ verschiedene MAC-Adressen generieren.


\paragraph{Normen für Ethernet}
Als Standards werden hauptsächlich \emph{IEEE 802.3} und \emph{Ethernet2} verwendet.


\begin{center}
\includegraphics[width=12cm]{img/05_Ethernet_Rahmen.png}
\end{center}

Die Definition eines Ethernet-Rahmens impliziert, dass sie eine MTU (1500 Bytes) und eine Minimum Transmission-Unit (46 Bytes).

Anhand Byte 13 und 14 kann unterschieden werden nach welchem Standard, der Rahmen aufgebaut ist:
\begin{itemize}
\item Inhalt $>$ 0x0600: Ethernet2
\item Inhalt $<$ 0x0600: IEEE 802.3
\end{itemize}

Das Netzwerk-Interface prüft bei jedem Rahmen, wie er interpretiert werden muss. Bei IEEE-Rahmen folgt auf das Feld Length/Type das Feld LLC (3 Byte). LLC enthält den Protokollcode für das innere Protokoll der Payload. Alle IEEE 802.n-Normen haben das gleiche LLC-Feld.



\paragraph{Kanalzugriffsverfahren für Ethernet}
Carrier Sense Multiple Access / Collision Detection: Eine Station, die senden möchte, hört zuerst, ob der Kanal frei ist. Wenn Kanal frei ist, darf gesendet werden. Dabei kann es zur Kollision kommen.

Eine Kollision muss sicher von jeder Station detektiert werden. Bei einer Kollision: \emph{1)} Senden des Jam-Signals; \emph{2)} Rückzug vom Kanal; \emph{3)} Station generiert Zufällige $n$ und wartet $n*t_s$ wobei $t_s$ die Slotzeit ist.
Dafür muss jeder Ethernet-Rahmen eine minimale Länge haben.

\paragraph{Repeater und Bridge}
Ein \emph{Repeater} ist ein elektrischer Verstärker in einem "shared medium". Segment 1 und 2 bleiben Verbunden und bilden somit eine Kollisionsdomäne.

Eine \emph{Bridge} macht Zwischenspeicherung von Rahmen in Layer 2. Somit wird in zwei Kollisionsdomänen aufgetrennt.

\emph{Multiport Bridge} -- Switch: Auftrennung in verschiedene Kollisionsdomänen.


\begin{center}
\includegraphics[width=12cm]{img/05_Bridge.png}
\end{center}


\paragraph{MAC-Tabelle} -- Lernen der MAC-Adressen: Switch merkt sich für jeden ankommenden Rahmen den \emph{Eingangsport} und die \emph{Absenderadresse}. Die Absenderadresse wird in die entsprechende Zeile eingefüllt.
Nicht mehr auftretende MAC-Adressen werden wieder gelöscht.

Anhand der Tabelle werden ankommende Rahmen weitergeleitet.

\paragraph{Hub und Switch} -- Weiterleitung von Ethernet-Rahmen

Auf Hub muss geprüft werden, ob Bus frei, dann kann gesendet werden. Wenn ein Switch verwendet wird, können alle senden und empfangen -- Switch handelt den Verkehr.

Ein Switch hat vier Operationen: \emph{1)} Learning (MAC-Tabelle), \emph{2)} Aging (nach ca. 2 Min ohne Auftreten wird ein Eintrag gelöscht), \emph{3)} Flooding, \emph{4)} Selective Forwarding \\

\emph{Flooding} -- Wenn Zieladresse von ankommendem Rahmen nicht in MAC-Tabelle: Rahmen wird an alle Ports (ausser Eingansport) weitergeleitet.

\emph{Selective Forwarding} -- Zieladresse ist in MAC-Tabelle: Rahmen wird nur an den entsprechenden Port weitergeleitet.

\newpage
\paragraph{Ablauf ARP}
Wenn der Quellrechner die Ziel-Adresse, aber nicht die Ziel-MAC-Adresse kenn wird die Zuordnung der MAC-Adresse zu einer IP-Adresse mittels \emph{Address Resolution Protocol} gemacht. Rechner und Router speichern die MAC-Adresse con Ziel-IP-Adressen lokal in der ARP Tabelle. Nach entsprechender Zeit werden die Einträge in der Tabelle entfernt (gealtert).

\emph{Probleme} -- Zu viel Broadcastverkehr; ARP-Spoofing (Attacker gibt sich als Default Gateway aus und kann so allen Verkehr mitlesen)


\newpage
\section{Netzwerkschicht} 
Von einem Netz ins andere und Protokolle IPv4, IPv6, AppleTalk, ICMP, OSPF und andere werden angewendet. Daten von Schicht 2 werden in Schicht 3 gekapselt.

\paragraph{Aufgaben Schicht 3}
\begin{itemize}
\item Adressierung
\item Kapselung in Senderichtung
\item Wegleitung -- Routing durch viele Netze
\item Entkapselung beim Empfang
\item Fehlerbehandlung
\end{itemize}

\paragraph{Eigenschaften IP}
Das Protokoll \emph{IP} ist:
\begin{itemize}
\item \emph{verbindungslos} -- Es werden IP-Pakete bei der Quelle los geschickt ohne, dass das Ziel davon weiss
\item \emph{best effort} -- Das Netz leitet soviele Pakete weiter, wie gerade möglich; Pakete werden weggeworfen bei Überlauf
\item \emph{Media independent} -- Läuft über allen möglichen Schicht 2 Protokollen (Ethernet, TokenRing, PPP, ATM) und Schicht 1 Medien
\end{itemize}



\paragraph{Header: IPv4} -- Der Header wird mit Zeilen à vier Byte dargestellt.

\begin{itemize}
\item \emph{Version} -- IP Version 4
\item \emph{IP Header Length} -- Header hat nicht immer Länge von 20 Bytes; wird in Anzahl Zeilen, $n*4$ Bytes angegeben
\item \emph{Differentiated Services} -- ursprünglich vorgesehen, Verkehr zu priorisieren -- wird nicht angewendet; kann in lokalen Services für QoS für e.g. Sprachanwendungen verwendet werden
\item \emph{Total Length} -- Länge des gesamten Paketes
\item \emph{Identification, Flag, Fragment Offset} -- werden (selten) verwendet, wenn Paket grösser als zulässige länge für Schicht 2 Protokoll (e.g. 1500 Bytes für Ethernet); Oft machen Implementationen der Transport Schicht (Layer 4) Segmente mit einer maximalen Länge von 1460 Bytes;
\item \emph{Time-to-live} -- Zahl dekrementiert bei jedem Hop; wenn TTL=0 wird das Paket verworfen und über ICMP Meldung an Absender; wird benötigt, damit Pakete nicht unendlich lange kreisen
\item \emph{Protocol} -- gibt an, welchem Protokoll ein angekommenes IP-Paket übergeben werden soll
\item \emph{Header Checksum} -- Paket wird verworfen, wenn Checksum nicht korrekt
\item \emph{Source IP Address} -- Adresslänge 4 Bytes
\item \emph{Destination IP Address} -- Adresslänge 4 Bytes
\item \emph{Options} -- wird oft nicht verwendet;
\item \emph{Padding} 
\end{itemize}

\paragraph{Header: IPv6}
\begin{itemize}
\item \emph{Version} -- Inhalt ist 6
\item \emph{Traffic Class (4 bit)} -- ursprünglich vorgesehen, Verkehr zu priorisieren -- wird nicht angewendet; kann in lokalen Services für QoS für e.g. Sprachanwendungen verwendet werden
\item \emph{Flow Label (20 bit)} -- Spezial-Dienst für Echtzeitanwendungen; Information an die Router, den selben Pfad für den Strom beizubehalten; Pakete müssen an Ziel nicht geordnet werden
\item \emph{Payload Length (16 bit)} -- Länge des ganzen IP-Paketes; exklusive Header und Extension Header
\item \emph{Next Header (8 bit)} -- Gibt das Protokoll an, das auf den IPv6-Header folgt
\item \emph{Hop Limit (8 bit)} -- Wird bei jedem Router dekrementiert. Erreicht es $0$ wird das Paket verworfen und es wird eine ICMPv6 Meldung an Quelle gesendet (Paket nicht an Ziel angekommen)
\item \emph{Source IP Address (128 bit)}
\item \emph{Destination IP Address (128 bit)}
\end{itemize}


\paragraph{Was macht ein Router mit einem eingehenden Paket?} -- Paket wird anhand der Routing-Tabelle weitergeleitet



\paragraph{Funktion von Default Gateway} -- Ein Default Gateway muss an jedem Rechner (der nach aussen kommunizieren soll), auf jedem Switch (der aus anderen Netzen angesprochen werden soll) und auf jedem Router konfiguriert werden.

Das Default Gateway wird beim Rechner entweder manuell oder per DHCP-Server definiert.\\Auf Switch: \verb+S1(config)#ip default-gateway 192.168.10.1+



\paragraph{Funktion von Routing-Tabelle} -- Es kann zwischen Routing-Tabelle auf Rechner und Router unterschieden werden. 

In der Routing-Tabelle auf dem Rechner wird das Loopback Interface (127.0.0.1), das lokale Netz und eine Default-Route eingetragen. Für die Default-Route muss das Default Gateway konfiguriert sein. Wird entweder durch DHCP Server oder manuell gesetzt.
 
In der Routing-Tabelle auf einem Router werden direkt angeschlossene Netze, entfernte Netze (statische Einträge; dynamische Routingprotokolle) und die Default Route eingetragen





\paragraph{Aufbau Router; Prozess während Aufstarten}
Ein Router hat:
\begin{itemize}
\item CPU -- Weiterleitung Pakete; rechnet Routing-Algorithmus
\item RAM -- IOS wird nach Aufstarten in RAM geladen; enthält running-config; enthält Routing-Tabelle; enthält ARP-cache
\item ROM -- Diagnostic Software für Tests bei Aufstarten (POST: Power On Self Test); Speichert bootstrap BefehleM
\item Flash -- Permanente Speicherung IOS
\item NVRAM -- nichtflüchtiger Speicher für startup-config
\item mindestens 2 Netzwerk-Interfaces -- Anschlüsse
\end{itemize}

Beim Aufstarten wird \emph{1)} POST durchgeführt, \emph{2)} die bootstrap Befehle geladen, \emph{3)} das IOS lokalisiert und geladen, \emph{4)} und das configuration file lokalisiert und geladen.

Die Konfiguration wird entweder aus dem NVRAM oder von einem TFTP-Server geladen.



\paragraph{Zustände bei Router abfragen}
\begin{lstlisting}
Router#show ip interface brief // IF Zustand
Router#show ip route // Routing-Tabelle
\end{lstlisting}


\newpage
\section{IP-Addressierung}

\paragraph{IPv4: Adresse und Subnetzmaske}

\begin{itemize}
\item \emph{Netzadresse} -- Hat lauter Nullen im Hostteil (erste Adresse des Netzes)
\item \emph{Broadcastadresse} -- Hat lauter Einsen im Hostteil (letzte Adresse)
\item \emph{Hostdressen} -- alles zwischen Netz- und Broadcastadressen
\end{itemize}

Rechner- und Router-IFs erhalten \emph{immer} eine Hostadresse.

 

\paragraph{IPv4: Unicast, Multicast und Broadcast} -- Es gibt drei verschiedene Arten von Ziel-Adressen:
\begin{itemize}
\item \emph{Unicast} -- Ein IP-Paket geht an genau ein IF. In einem /24-er Netz sind dies die Adressen 1 bis 254
	\begin{itemize}
	\item Werden auf Rechner entweder über DHCP-Server oder manuell eingerichtet
	\end{itemize}
\item \emph{Broadcast} -- Ein Rechner stellt eine Anfrage mit lauter Einsen im Hostteil (255)
\item \emph{Multicast} -- Pakete an eine Gruppe von Rechnern senden; 224.0.0.0 - 239.255.255.255; 224.0.0.0 - 224.0.0.255 nur link-lokal und werden nicht geroutet; 224.0.1.0 - 239.255.255.255 sind globale Multicast Adressen
	\begin{itemize}
	\item Hilft bei Verteilung eines Datenstroms; Sender sendet Pakete nur einmal und Switch/Router leiten Pakete an jeweilige Ports weiter
	\end{itemize}
\end{itemize}

\paragraph{Unterscheiden: Öffentliche und Private Adressen} -- Es sind drei Adressräume für private Verwendung vorgesehen: 10.0.0.0/8; 172.16.0.0/12; 192.168.0.0/16

Private Adressen werden im öffentlichen Internet nicht weitergeleitet.

\paragraph{Spezielle Adressen}
\begin{itemize}
\item 127.0.0.1 -- Loopback (127.0.0.1/8 ist reserviert)
\item 169.254.0.0/16 -- Link-lokale Adressen; können automatisch dem "Local Host" zugewiesen werden
\item 192.0.2.0/24 -- Unterricht und Dokumentation
\item 240.0.0.0/4 -- ist reserviert und darf nicht gebraucht werden
\end{itemize}
	
	
\paragraph{Klassenbezogene und Klassenlose Adressierung} -- Ursprünglich wurde der IPv4 Adressbereich in fünf Klassen unterteilt (A-E). Mit der Netzklasse wird nicht die tatsächliche Grösse eines Netzes angegeben, sondern wie viele Adressen es umfassen kann.
\begin{itemize}
\item A -- 0.0.0.0 - 127.255.255.255; 0 + 7 bit Netz, 24 bit Host;128 Netze; 16'777'214 Hosts pro Netz
\item B -- 128.0.0.0 - 191.255.255.255; 10 + 14 bit Netz, 16 bit Host; 16'384 Netze; 65'534 hosts
\item C -- 192.0.0.0 - 223.255.255.255; 110 + 21 bit Netz, 8 bit Host; 2'097'150 Netze; 254 Hosts pro Netz
\item D -- 224.0.0.0 - 239.255.255.255 (Multicast Gruppe); 1110 + 28 bit Multicast-Gruppen-ID
\item E -- 240.0.0.0 - 255.255.255.255 (reserviert; wird nicht genutzt); 1111 + 28 bit
\end{itemize}

Mittlerweile ist diese Technologie veraltet und es wird mit Subnetzmasken definiert, welchen Adressbereich ein Netz abdeckt. Dies nennt man klassenlos oder Classless Inter-Domain Routing (CIDR).

\paragraph{IPv6: Adressierung, Struktur mit Präfix. Subnet-ID und Interface-ID}
Mit der Einführung von IPv6 wird \emph{1)} ein grösserer Adressraum eingeführt  \emph{2)} hierarchische Vergabe der Adressen unterstützt \emph{3)} eine feste Headerlänge definiert  \emph{4)} das ICMP verbessert und  \emph{5)} Network Address Translation (NAT) abgeschafft.

IPv6-Adressen bestehen aus einem Prefix und einer Interface ID. Diese sind definiert durch "/n" nach der Adresse, wobei es n-bit des Prefixes definiert. 2001:0DB8:000A::/64 entspricht also 2001:0DB8:000A:0000 (prefix) :0000:0000:0000:0000 (Interface ID)

\paragraph{Übergang IPv4 zu IPv6} -- Mit tunneling von IPv6-Inseln über ein IPv4-Netz können IPv6-Netze miteinander sprechen.

\paragraph{Dual-Stack} ist, wenn IPv4 und IPv6 koexistieren. Dabei laufen beide Protokolle parallel. Wenn beide Enden IPv6 unterstützen, läuft Kommunikation über IPv6, ansonsten über IPv4.

\paragraph{IPv6 Adressen}
\begin{itemize}
\item \emph{Unicast} -- Ein Empfänger
	\begin{itemize}
	\item Global -- 2000::/3; weltweig eindeutig; Im Internet geroutet
	\item Link-Local -- weden nur lokal verwendet; FE80::/10
	\item Loopback -- Logisches IF zum eigenen IPv6 Stack; ::1/128
	\item Unspecified -- kann verwendet werden, wenn Quelle irrelevant ist
	\item Unique/Site Local -- Nicht verwenden für Kommunikation ins Internet; Für Kommunikation in Firmennetzen
	\item Embedded -- IPv4 kompatible Adressen für IPv4-IPv6 Translation
	\end{itemize}
\item \emph{Multicast} -- eine Gruppe von Empfängern
\item \emph{Anycast} -- Eine unicast Adresse, die Gruppen von Network IFs zugewiesen wird; Das Paket wird an das nächste IF weitergeleitet
\end{itemize}

\paragraph{IPv6 Unicast Adressen}
\begin{itemize}
\item \emph{Global Routing Prefix} -- Regional Internet Registry vergibt Adressen aus dem Adressraum 2000::/3; ISP erhält /32er-Adressbereiche; ISP geben Kunden /48er oder kleinere Netze;
\item \emph{Subnet ID} -- Adress-Teil vom 49. bis zum 64. bit
\item \emph{Interface ID} -- 64 letzte bit; entsprechen Host-Teil von IPv4; Rechner kann mehrere IPv6-Adressen auf einem physikalischen IF haben
\end{itemize}

Auf Cisco Router muss IPv6-Routing eingeschaltet werden:
\begin{lstlisting}
// IPv6 Routing einschalten
R1(config)#ipv6 unicast-routing 

// Interface fa0/0 mit IPv6 Adresse konfigurieren
R1(config)#interface fa0/0
R1(config-if)#ipv6 address 2001:db8:acad:1::1/64
R1(config-if)#no shutdown

// Kontrolle
R1#show ipv6 interface brief
R1#show ipv6 route // Routing Tabelle
    // Spezifische Adresse pingen (Verbindung testen)
R1#ping 2001:db8:acad:10::5
\end{lstlisting}

\paragraph{ICMPv4/v6} -- Mit dem Internet Control Message Protocol kann überprüft werden, ob in einem Netzwerk das Routing soweit stimmt, dass Rechner A den Rechner B in einem anderen Netz erreichen kann.

Die wichtigsten ICMP Meldungen sind:
\begin{itemize}
\item Host confirmation mit echo request und echo response -- feststellen ob ein Host erreichbar ist (ping, traceroute)
\item Destination or Service Unreachable -- Router sendet Nachricht an Absender, wenn er ein Paket nicht weiterleiten kann, weil er keinen passenden Eintrag in der Routingtabelle findet.
\item Time exceeded -- TTL auf null; Fehlermeldung zurück an Absender
\item Route redirection -- Router kann einem direkt angeschlossenen Rechner sagen, dass es einen schnelleren Weg gibt als über diesen Router
\end{itemize}

Protokoll-Stack von ICMPv4 ist [L2-Header $|$ IP $|$ ICMP] -- L2 Header ist meistens ein Ethernet Header; ICMP ist Schicht-3-Protokoll (benützt Dienste von Schicht 3 aber nicht höhere).

Bei ICMPv6 wird zudem \emph{1)} Router Solicitation Meldungen (RS),  \emph{2)} Router Advertisments (RA) \emph{3)} Neighbour Solicitation (NS) und \emph{4)} Neighbour Advertisment (NA) durchgeführt.
RS und RA werden für die Autokonfiguration bei IPv6 verwendet. NS und NA werden für Address resolution (IPv4 ARP) und Duplicate Address Detection verwendet.

\emph{Neighbour Solicitation für Adress Resolution} -- Station sendet eine ICMPv6 NS Meldung zu einer IPv6 Adresse, um zugehörige MAC-Adresse zu finden.

\newpage
\section{Unterteilen und Zusammenfassen von IP-Netzen}
Da der IPv4-Adressraum rasch knapp wurde, hat man begonnen Adressraum wo möglich zu sparen. Netze wurde nur so gross gemacht, wie nötig -- man hat begonnen IPv4-Netze zu unterteilen.

Zudem ist es sinnvoll, nicht zu grosse Netze zu haben. Jede Station verursacht Broadcast-Anfragen. Viele Hosts in einem Netz machen viel Broadcast-Verkehr.

Mit der Unterteilung von IPv4-Netzen kann dem entgegengewirkt werden.

\paragraph{IPv4-Netz in gleich grosse Subnetze unterteilen} -- Netze können halbiert werden, um in gleich grosse Subnetze unterteilen.
Beispiel: 192.68.1.0/24 in zwei Netze
\begin{itemize}
\item 192.68.1.0/25
\item 192.68.1.127/25
\end{itemize}
Somit wurde das ursprüngliche Netz in zwei gleich grosse Subnetze unterteilt. Anzahl Subnetze pro Netz ist jeweils eine 2er-Potenz.

Anzahl nutzbare Host-Adressen ist $2^{32-n} - 2$ -- also 2er-Potenz minus Netzadresse und Broadcast-Adresse

\paragraph{IPv4-Netz in verschieden grosse Subnetze unterteilen} -- Für WAN-Netze werden /30er Netze verwendet! Diese werden oft ans Ende des Adressraumes gesetzt.


\paragraph{Berechnungen mit Subnetzmasken variabler Länge}
Geeignete Subnetzmasken bei variabler Grösse:
\begin{itemize}
\item Bis 126 Hosts -- 255.255.255.128
\item Bis 62 Hosts -- 255.255.255.192
\item Bis 30 Hosts -- 255.255.255.224
\item Bis 14 Hosts -- 255.255.255.240
\item Bis 6 Hosts -- 255.255.255.248
\item Bis 2 Hosts -- 255.255.255.252 (kann für WAN verwendet werden)
\end{itemize}

\setlength{\tabcolsep}{10pt}
\renewcommand{\arraystretch}{1.2}


Beispiel:
\begin{center}
\begin{tabular}{|m{2cm}| m{0.5cm} |m{2.8cm}|m{2cm}|m{2cm}|} 
Netzname & \#  & Subnetzmaske & Netzadresse & Broadcast-Adresse\\
\hline
Sales Office & 40 & 255.255.255.192 & 172.16.0.0 & 172.16.0.63\\
Technical Support & 35 & 255.255.255.192 & 172.16.0.64 & 172.16.0.127\\
Engineering & 30 & 255.255.255.192 & 172.16.0.128 & 172.16.0.191\\
HR & 23 & 255.255.255.224 & 172.16.0.192 & 172.16.0.223\\
Executive Mgmt & 10 & 255.255.255.240 & 172.16.0.224 & 172.16.0.239\\
WAN1 & - & 255.255.255.252 & 172.16.0.240 & 172.16.0.243\\
WAN2 & - & 255.255.255.252 & 172.16.0.244 & 172.16.0.247\\
WAN3 & - & 255.255.255.252 & 172.16.0.248 & 172.16.0.251\\
WAN4 & - & 255.255.255.252 & 172.16.0.252 & 172.16.0.255\\
\end{tabular}
\end{center}

\setlength{\tabcolsep}{18pt}
\renewcommand{\arraystretch}{1.2}

Daraus resultieren die Netze:
\begin{itemize}
\item 172.16.0.0/26
\item 172.16.0.64/26
\item 172.16.0.128/26
\item 172.16.0.192/27
\item 172.16.0.224/28
\item 172.16.0.240/30
\item 172.16.0.244/30
\item 172.16.0.248/30
\item 172.16.0.252/30
\end{itemize}


\paragraph{Netze in IP-Adressräumen zusammenfassen}
Die Zusammenfassung muss den gleichen Adressraum bedecken, wie die einzelnen Netze.


\paragraph{Adressraum sinnvoll einteilen und planen}

Zudem wird empfohlen bei der Belegung eines Adressraumes nach einem ähnlichen Muster zu folgen. Beispiel:
\begin{center}
\begin{tabular}{|m{4cm}| m{1cm} |m{1cm}|} 
Engerät & Von & Bis\\
\hline
Router / Switch & .1 & .15\\
Andere Netzwerkgeräte & 16 & .23\\
Server & .24 & .31\\
Clients & .32 & .240\\
Reserve & .241 & .255\\
\end{tabular}
\end{center}

\paragraph{Entwurfsleitlinien für IPv6} -- IPv6-Netze immer gleich gross wählen: /64

\newpage
\section{Transportschicht}
Schicht 4 sorgt dafür, dass der Sendet, so langsam sendet, wie die langsamste Leitung senden kann.

\paragraph{Aufgaben der Transportschicht (Schicht 4)} 
\begin{itemize}
\item Multiplexierung
\item Falls grosse Datenblöcke erwartet werden:
    \begin{itemize}
    \item Segmentierung und Wiederzusammensetzen von grossen Datenblöcken
    \item Sicherung der Übertragung (fehlerfreie Übertragung)
    \item Flusssteuerung
    \end{itemize}
\end{itemize}

\emph{Multiplexierung} -- Die Datenströme werden beim Empfänger entsprechend der Anwendung zugeordnet. Gewisse Anwendungen (e.g. für Filetransfer) senden / empfangen grosse Datenblöcke.

\emph{Segmentierung} -- Wenn ein Datenblock grösser als die MSS (Maximum Segment Size) ist, segmentiert das Transport-Protokoll die Daten und verpackt jedes Segment einzeln in ein IP-Paket. Segmente werden beim Zielrechner wieder zusammengesetzt.

\paragraph{Anforderungen von Anwendungen an die Kommunikation} Es gibt Anwendungen, welche grosse und andere die kleine Datenblöcke senden. Dafür werden zwei Protokolle unterschieden:


\paragraph{Protokolle in Schicht 4}
Am verbreitetsten für Schicht 4 sind die Protokolle TCP und UDP.


\emph{TCP} -- Verbindungsorientierte, fehlerfreie Übertragung; viel Overhead -- kann also langsam werden; Anwendungen benötigen einen zuverlässigen Dienst

\emph{UDP} -- Anwendungen mit Nachrichten kürzer als die MSS; verbindungslos -- schnell; Jedes Datagramm wird am Ziel einzeln und sofort der Anwendung abgeliefert;


\begin{center}
\includegraphics[width=12cm]{img/09_protocols.png}
\end{center}

\subsection{TCP}
TCP baut eine bidirektionale (verbindungsorientierte) Verbindung zwischen Client und Server auf. Dabei werden zwei Datenströme (A $\leftarrow$ B und A $\rightarrow$ B) vorbereitet und kontrolliert.

TCP segmentiert grosse Blöcke, nummeriert sie, bestätigt empfangene Segmente, wiederholt nicht bestätigte Segmente, ordnet die Reihenfolge empfangener Segmente und regelt die Geschwindigkeit der Übertragung. 


Wenn TCP verwendet wird, wird -- gesteuert durch das Feld \emph{Control} -- eine bidirektionale Verbindung aufgebaut. Dabei werden \emph{1)} die Maximum Segment Site (MSS), \emph{2)} Selective Acknowledgement (Ja/Nein) und \emph{3)} die Skalierung der Window-Size vereinbart.

\paragraph{TCP (Header und Bedeutung der Felder)}
\begin{itemize}
\item Source Port (16 Bit) -- Bezeichnung der Anwendung auf der Quellrechner
\item Destination Port (16 Bit) -- Bezeichnung der Anwendung auf der Empfangsrechner
\item Sequence Number (32 Bit) -- Zählen der Bytes, welche gesendet werden
\item Acknowledge Number (32 Bit) -- Zählen der Bytes, die empfangen werden und Bestätigung bei ACK
\item Header Length (4 Bit) -- Länge des Headers
\item Reserved (6 Bit) -- 
\item Control Bits (6 Bit) -- 6 Bits für 6 verschiedene Flags: Urgent (URG), Acknowledgement (ACK), Push (PSH), Reset (RST), Synchronize Sequence Numbers (SYN), Finish (FIN)
\item Window (16 Bit) -- gibt an, wie viele Bytes eine Station senden darf, bevor von der Gegenseite der Empfang der Daten bestätigt wurde; Wird gegenseitig zugesprochen
\item Checksum (16 Bit) -- 
\item Urgent (16 Bit) -- 
\item Options (0 or 32 Bit) -- 
\end{itemize} 

\paragraph{TCP Header Flags im Feld Control}
\begin{itemize}
\item \emph{URG} -- Urgent (wird nicht mehr benützt)
\item \emph{ACK} -- Acknowledgement: Ist dann gesetzt, wenn die Verbindung
geöffnet ist.
\item \emph{PSH} -- Push: Zeigt das letzte Segment eines Datenblocks an. Wenn ein Empfänger ein Segment mit gesetztem PSH erhält, so setzt er die Segmente zusammen und übergibt den Datenblock der Anwendung.
\item \emph{RST} -- Reset
\item \emph{SYN} -- Synchronize Sequence Numbers (siehe unten)
\item \emph{FIN} -- No more data from sender
\end{itemize}

\paragraph{Verbindungsaufbau TCP} Für den Verbindungsaufbau wird ein Drei-Wege-Handshake ohne Payload angewendet (send nach received):
\begin{enumerate}
\item Send SYN (Sender $ \rightarrow $ Empfänger)
\item Send SYN, ACK (Sender $ \leftarrow $ Empfänger)
\item Send  ACK (Sender $ \rightarrow $ Empfänger)
\end{enumerate}

Danach ist eine Verbindung etabliert.


\begin{center}
\includegraphics[width=12cm]{img/09_tcp_handshake.png}
\end{center}

\paragraph{Verbindungsabbau TCP} Für den Verbindungsabbau werden vier Schritte benötigt (send nach received):
\begin{enumerate}
\item Send FIN (Sender $ \rightarrow $ Empfänger)
\item Send ACK (Sender $ \leftarrow $ Empfänger)
\item Send  FIN (Sender $ \leftarrow $ Empfänger)
\item Send  ACK (Sender $ \rightarrow $ Empfänger)
\end{enumerate}

Die etablierte Verbindung wird entweder so abgebaut oder es findet ein Time-Out statt.

\paragraph{Schwachstellen TCP} Wenn der Client (Sender) den dritten Schritt nicht sendet, bleiben beim Server (Empfänger) ggf. RAM-Ressourcen reserviert. Wenn viele Clients dies machen, dann kann der Server langsam werden.





\subsection{UDP}
macht keine Flusskontrolle
\paragraph{UDP (Header und Bedeutung der Felder)}







\paragraph{Fehlerkorrektur} Wenn ein Segment nicht ankommt sendet der Empfänger 3-mal eine ACK. Somit wird dem Sender gezeigt, dass dieses Segment erneut gesendet werden soll. Der Empfänger fordert jedes Segment an, welches nicht in der erwarteten Reihenfolge ankommt -- \emph{Fast Retransmission}.

\paragraph{Flusssteuerung / Flusskontrolle} Die Window-Size wird jedesmal halbiert, wenn ein Segment nicht angekommen ist. Somit wird der Fluss gesteuert, damit die Netze nicht überlastet werden. Wenn Segmente wieder durchkommen, wird die Window-Size wieder vergrössert.

Flusskontrolle wird in UDP nicht gemacht.

\paragraph{Vorteile und Gefahren UDP} Da der Header von UPD sehr klein ist, entsteht wenig Overhead. Somit können kurze Nachrichten effizient und schnell ausgeführt werden.

Beim UDP-Empfänger wird jedes empfangene Datagramm sofort an die Anwendung geliefert.


Damit e.g. VoIP funktioniert, wird für die Sprache das Real-Time-Protocol (RTP) verwendet. Damit werden die Datagramme nummeriert.


\paragraph{Abschluss}
UDP ist gefährlich und TCP ist gutmütig dafür langsam.





\section{Anwendungsschicht}
Anwendungen übernehmen die Aufgaben der theoretischen Schichten \emph{Darstellung} und \emph{Sitzung}.

\subsection{Position der Anwendung innerhalb des Protokoll-Stacks}
Die Anwendung 
\begin{itemize}
\item ist die oberste Schicht im Protokollstack und stösst den Datenaustausch an -- ist der Beginn der Kommunikation
\item hat ganz eigene Regeln, wie zwei Endsysteme Daten miteinander austauschen (e.g. HTTP -- geht immer über TCP)
\item greift über einen "Socket" auf die Transportschicht zu. Socket: Über das Vier-Tupel [Quell-IP-Adresse, Quell-Port-Nr., Ziel-IP-Adresse, Ziel-Port-Nr.] (ist der Socket) kann eine Verbindung eindeutig bestimmt werden.
\end{itemize}


\subsection{Standards der Internetanwendungsprotokolle}

\paragraph{HTTP} geht über TCP und Port 80; Möglicher Protokollstack: [HTTP; TCP: IP; Ethernet]; Ablauf, wenn alle Speicher leer sind:
\begin{itemize}
\item Ermittlung der MAC-Adresse des Default Gateway: ARP-Anfrage für IPv4 oder Neighbour Discovery mit ICMPv6
\item Auflösung des Domainnames in eine IP-Adresse mit der Anwendung DNS
\item Verbindungsaufbau mit dem TCP-3-Weg-Handshake von Endgerät zu Endgerät
\item Datenaustausch
\item Beim Schliessen des Browsers: Abbau der TCP-Verbindung
\end{itemize}

\paragraph{eMail}
Möglicher Protokollstack: [SMTP/POP3; TCP: IP; Ethernet]
\begin{itemize}
\item SMTP -- Simple Mail Transfer Protocol (port 25)
\item POP3 -- Post Office Protocol (port 110)
\item IMAP -- Internet Message Access Protocol (port 143)
\end{itemize}

Wenn die Mail-Server in der Cloud sind erfolgt der Nachrichtenaustausch über https.

\paragraph{DHCP} Dynamic Host Configuration Protocol; DHCP Server port 67; DHCP Client port 68; Möglicher Protokollstack: [DHCP; UDP: IP; Ethernet] (UDP, da keine IP-Adresse bekannt);


\paragraph{FTP}

\subsection{Ansätze DNS}
Wird benötigt für die Zuordnung von Servernamen zu IP-Adressen und umgekehrt. Die DNS löst einen Full Qualified Domain Name (FQDN) in eine IP-Adresse auf.

\begin{itemize}
\item ist dezentral und hierarchisch (delegiert nach unten) -- Root-Level Domain (Server); Top-Level Domain (Server); Second Level Domain (Server)
\item beruht auf Delegation
\item Die IP-Adresse des Servers, bestimmt dessen Betreibers
\end{itemize}

Verschiedene Resource records eines DNS Servers:
\begin{itemize}
\item A -- IPv4 Adresse
\item AAAA -- IPv6 Adresse
\item NS -- Angabe des authorativen Name Servers
\item MX -- Mail exchange record
\item CNAME -- Alias
\item SOA -- Start Of zone Authority
\item PTR -- Domain Name Pointer (inverse Operation)
\item TXT -- TeXT string (gefährlich; Datenexfiltration)
\end{itemize}

\paragraph{Angriffspunkte} Die Antworten werden nicht auf Authentizität geprüft und könnten gefälscht werden; Mit DNSSEC und DNS over HTTPS gibt es aber neue, sicherere Varianten von DNS.


\paragraph{Ablauf DNS}
\begin{enumerate}
\item
\end{enumerate}



\section{Routing-Konzepte}

\subsection{Aufgaben eines Routers}
Der Router hat zwei Aufgaben: \emph{1)} Finden des besten Weges zu den Zielnetzen und \emph{2)} Weiterleitung der Pakete.

\paragraph{Bestimmung des besten Weges} In vermeshten Netzen wird meist ein dynamisches Routing Protokoll ausgeführt.
\begin{itemize}
\item Ein Routing Protokoll benötigt eine \emph{Metrik}, um den Weg der Pakete zu bestimmen
\item Metriken:
  \begin{itemize}
  \item Anzahl Hops (RIP)
  \item Summe der Kosten der einzelnen Links (konstante / Datenrate) OSPF; Die Kosten  eines Links sind proportional zur Bandbreite des Links
  \end{itemize}
\end{itemize}

\subsection{Longest Prefix Matching}
Die Einträge in der Routing Tabelle sind IP-Netze. 
Es wird der Eintrag mit der längsten Subnetzmaske gewählt; der spezifischste Eintrag. Dies bedeutet, dass immer die gesamte Routing Tabelle abgesucht werden muss.


\begin{center}
\includegraphics[width=12cm]{img/11_lpm.png}
\end{center}


\subsection{Paketweiterleitung}

\subsection{Routing-Tabellen}
Schritte der Paketweiterleitung:
\begin{enumerate}
\item Entfernung des L2-Rahmens beim eingehenden Paket
\item Finden der Zieladresse im IP-Header
\item Absuchen der Routing Tabelle und Bestimmung des Ausgangs-Interfaces
\item Entweder Weiterleitung an Zielgerät (Bildung eines neuen Ethernet Headers) -- oder Weiterleitung an den Next-Hop
\item Paket wird verworfen, falls kein zutreffender Eintrag in der Routing Tabelle
\end{enumerate}


\paragraph{Prinzip}
\begin{itemize}
\item Jeder Router trifft seine Entscheidungen für die Weiterleitung selbstständig aufgrund seiner Routing Tabelle. R1 weiss nicht was bei R2 in der Tabelle steht
\item Die Einträge in der Routing Tabelle von R1 sind nicht notwendigerweise konsistent mit R2
\item Information über den kürzesten Weg beinhaltet nicht, dass alle Router auf dem Weg den Rückweg kennen
\end{itemize}

Der Weg zu den entfernten Netzen muss entweder durch statische Routen oder durch ein dynamisches Routing Protokoll eingetragen werden.\\

Beiträge in der Routing Tabelle:
\begin{itemize}
\item \emph{Direkt angeschlossene Netzt} -- Alle an einen Router angeschlossenen Netze, deren Router-IF eine gültige IP Adresse konfiguriert haben und bei denen das Schicht-1/2 Protokoll läuft werden automatisch als \emph{directly connected routes} in die Routing Tabelle eingetragen
\item \emph{IP Adresse des lokalen IF} -- IP Adresse des lokalen Router IFs (/32 oder /128)\
\item \emph{Statische Routen} -- Werden durch den Administratoren manuell in die Routing Tabelle eingetragen
\item \emph{Statische Routen}
\end{itemize}

\subsection{Administrative Distanz}
Gibt es für ein Zielnetz mehrere Quellen für die Routinginformation, wird die Quelle mit der niedrigeren administrativen Distanz in Routing Tabelle eingetragen.

\begin{center}
\begin{tabular}{ | m{4.2cm} | m{3cm} | } 
Quelle der Route & Administrative Distanz\\ 
\hline
Direkt angeschlossen & 0\\
Statische Route & 1\\
EIGRP summary route & 5\\
External BGP & 20\\
Internal EIGRP & 90\\
IGRP & 100\\
OSPF & 110\\
IS-IS & 115\\
RIP & 120\\
External EIGRP & 170\\
Internal BGP & 200\\
\end{tabular}
\end{center}

\paragraph{Beispiele IPv4 und IPv6 Routing Tabellen}

\begin{center}
\includegraphics[width=12cm]{img/11_ipv4_route.png}
\end{center}



\begin{center}
\includegraphics[width=12cm]{img/11_ipv6_route.png}
\end{center}



\section{Statisches Routing}

\subsection{Informationen zum Weiterleiten von Paketen}
Ziel-Adresse, Quell-Adresse

\subsection{Vor- und Nachteile und Einsatzgebiete von statischem Routing}

\paragraph{Vorteile}
\begin{itemize}
\item Benötigen wenig Ressourcen
\item Routing ist stabil, sobald Routen definiert sind
\item Einfache Konfiguration in kleinen Netzwerken
\item Kein Routing Protokoll verwendet -- Weniger Angriffspunkte
\end{itemize}

Statisches Routing soll mit soviel Einträgen wie nötig, aber mit sowenig Einträgen wie möglich realisiert werden.

\paragraph{Nachteile}
\begin{itemize}
\item Aufwand für Unterhalt nicht zu unterschätzen
\item Durch falsche Konfiguration von Routen können Routing Loops entstehen
\item Änderungen an Topologie im Netzwerk müssen manuell auf jedem Router nachgepflegt werden
\end{itemize}


\paragraph{Einsatzgebiete} Firmennetze werden oft als Hub-and-Spoke Netze entworfen. Der Hub bekommt eine statische Route zu jedem Stub-Netz; Stub-Router erhält eine Default-Route.


\subsection{Stub-Netzwerk und Stub-Router}
Ein Netz, das nur einen Router und nur Weg zu einem Internetwork hat -- Wie Blatt am Baum; \emph{Stub-Router R1 erhält Default Route auf das WAN-Netz}; Router R2 erhält statische Route zum Stub-Netz.

\begin{center}
\includegraphics[width=12cm]{img/12_stub.png}
\end{center}


\subsection{Typen von statischen Routen}
\begin{itemize}
\item Standard statische Routen -- Bekanntgabe des Weges zu einzelnen Netzen
\item Default Routen -- Aller Verkehr zu unbekannten Zielen nimmt einen bestimmten Weg (IPv4: 0.0.0.0/0; IPv6: ::/0)
\item Floating Routen -- Angaben eines Backup Weges für den Fall, dass das Routing Protokoll ausfällt
\item Summary Routen -- Der Weg zu einer Reihe von Netzen wird zusammengefasst mit einer einzigen Wegangabe
\end{itemize}

\subsection{Next-Hop und IF-Routen}

\paragraph{Next-Hop}
Es werden Ziel-Netz (Netzadresse und Subnetzmaske) und IP-Adresse des next-hop angegeben. Spezifische Angabe von Empfänger, kann also in multi-access Netzen verwendet werden.

\verb+R2(config)# ip route 172.16.3.0 255.255.255.0 172.16.2.1+
\paragraph{IF-Routen}
Es werden Ziel-Netz (Netzadresse und Subnetzmaske) und der IF-Ausgang des next-hop angegeben. Da kein spezifischer Empfänger angegeben ist, wird ggf. ein Broadcast gesendet (also nur für P2P empfohlen, sonst können alle mitlesen).

\verb+R2(config)# ip route 172.16.3.0 255.255.255.0 ser0/0/0+

\paragraph{Wann welche anwenden}
Bei Multiaccess: Next-Hop; Bei P2P: IF-Routen.

\subsection{statische Routen für IPv4 und IPv6 in allen Varianten konfigurieren}
\begin{itemize}
\item \emph{Default Route}
    \begin{itemize}
    \item \emph{IPv4} -- \verb+R1(config)# ip route 0.0.0.0 0.0.0.0 ser0/0/0+
    \item \emph{IPv6} -- \verb+R1(config)# ipv6 route ::/0 s0/0/0+
    \end{itemize}
\item \emph{Summary Route}
    \begin{itemize}
    \item \emph{IPv4} -- \verb+Rx(config)#ip route 172.16.0.0 255.255.252.0 s0/0/0+
    \item \emph{IPv6} -- \verb+Rx(config)#ipv6 route 2001:db8:acad::/62 s0/0/0+
    \end{itemize}

\item \emph{Floating Static Route}
    \begin{itemize}
    \item \emph{IPv4} -- \verb+R1(config)#ip route 0.0.0.0 0.0.0.0 172.16.2.2 5+ (Administrative Distanz = 5)
    \item \emph{IPv6} -- \verb+R1(config)#ipv6 route ::/0 2001:db8:acad:4::2 111+
    \end{itemize}

\end{itemize}






\section{Dynamisches Routing}
Die Aufgaben sind \emph{1)} das Auflisten von entfernten Netzen, \emph{2)} die Laufende Aktualisierung der Routing-Information, \emph{3)} die Wahl des besten Weges zum Ziel, \emph{4)} das Finden eines neuen Weges zu einem Ziel im Falles eines Leitungsunterbruchs.\\

Datenstrukturen, Nachrichtenformate und Algorithmen werden definiert.

\subsection{Vor- und Nachteile von statischem Routing}


\paragraph{Vorteile}
\begin{itemize}
\item Skalieren besser
\item Automatisches Re-Routing bei Leitungsunterbüchen
\end{itemize}

\paragraph{Nachteile}
\begin{itemize}
\item i.A weniger sicher, weil Informationen in Band übers Netz ausgetauscht werden
\item Höhere Belastung von CPU, RAM und Leitungen
\item Wege können ändern
\item Eher komplexe Implementation
\end{itemize}



\subsection{Wann lohnt es sich dynamisches Routing einzusetzen}
Wenn sich die Netz-Struktur dynamisch ändert und regelmässig die Vorteile von dynamischem Routing relevant sind (e.g. Skalierbarikeit, Leitungsunterbrüche etc.)

\subsection{Autonomes System}
Ein \emph{autonomes System} (AS) ist eine Menge von Routern, die unter einer einheitlichen Verwaltung stehen. Typischerweise haben mittlere und grössere ISPs ein eigenes AS. Im allgemeinen Sprachgebrauch denkt man oft an eine Gruppe von Routern, die das gleiche Routing Protokoll benützen. Interior Gateway Protokolle (IGPs) sind Routing Protokolle, die innerhalb eines AS ausgeführt werden. Wesentlich komplexer sind Exterior Gateway Protokolle (EGPs), die zwischen AS Routen austauschen.


\subsection{Distanz-Vektor Protokolle}
Ein Router hat die Information, wie weit weg sich das Ziel in gegebener Richtung befindet. Dafür wird die Distanz-Vektor Technologie verwendet:
\begin{itemize}
\item DV Router sendet periodische Updates mit dem Inhalt seiner Routingtabelle
\item Updates werden per Broadcast oder Multicast gesendet; alle hören Broadcast aber nur TN, welche das Routing-Protokoll ausführen können etwas damit anfangen
\item DV Router kennt die Topologie des Netzes nicht; weiss nicht definitiv, wieviele Router im Netz mitmachen
\item Routing-Prozess wird dezentral durchgeführt; Berechnung Routing-Tabelle ist eine verteilte Anwendung
\end{itemize}

\paragraph{Algorithmus}
Der Algorithmus umfasst jeweils einen Mechanismus für \emph{1)} das Senden und Empfangen von Routing-Informationen, \emph{2)} die Berechnung des besten Weges zum Ziel und das Eintragen in die Routing-Tabelle und \emph{3)} das Entdecken von Änderungen in der Netz-Topologie und Massnahmen, um das Routing aktuell zu halten.

\paragraph{Vorteile}
\begin{itemize}
\item Sehr einfache Handhabung (Wenig Ausbildung nötig)
\item Benötigt wenig Rechnerleistung
\end{itemize}

\paragraph{Nachteile}
\begin{itemize}
\item Konvergiert nur langsam
\item Skaliert nicht
\item Neigt zu Routingschleifen
\end{itemize}



\paragraph{klassenbezogenes -- klassenloses Routing}
Wird in einem AS mit klassenbezogenen Netzen gearbeitet oder haben alle beteiligten Netze den gleichen Netzteil, so können klassenbezogene Routingprotokolle eingesetzt werden. Bei klassenbezogenen Routingprotokollen werden in den Updates die Netzadressen ohne Subnetzmasken ausgetauscht. Subnetze werden immer auf klassenbezogne Netze aufgerundet. So gibt der Router R1 in Abb. 13.6 dem Nachbarn R2 anstatt dem Subnetz 172.16.1.0/24 das zugehörige klassenbezogne Netz 172.16.0.0/16 bekannt.\\

RIPv1 und IGRP waren klassenbezogene Routing Protokolle. Nachfolgende Abbildung veranschaulicht die Problematik klassenbezogener Protokolle. Der Router R3 wird dem Router R2 ebenfalls das klassenbezogene Netz 172.16.0.0 bekannt geben.\\

Werden Netze unterteilt in Subnetze verschiedener Länge, so muss bei den Updates notwendigerweise zu jedem Netz die Netzmaske mitgegeben werden. Man spricht von klassenlosen Routingprotokollen. RIPv2 und EIGRP sind die die klassenlosen Varianten obiger klassenbezogener Protokolle.

\begin{center}
\includegraphics[width=12cm]{img/13_classless.png}
\end{center}


\subsection{RIP v2 für Firmennetz}
RIPv2 ist klassenlos. Als Metrik wird die Anzahl Hops verwendet (normalerweise begrenzt auf 15). Ist ein Netz weiter entfernt als die definierten Anzahl Hops, gilt es als nicht erreichbar.

Schritte zur Konfiguration:
\begin{enumerate}
\item Optional (sofern am lokalen Router angeschlossen -- R1): Statische Defaultrouten setzen
\item Routing Submenu wählen
\item RIP version 2 wählen und automatische Summarisierung ausschalten (\verb+no auto-summary+)
\item Bekanntgabe der Netze, die sich am Routing beteiligen sollen
\item Die Netze zu den Clients sollen passiviert werden; Es werden über die IF, an welchen clients angeschlossen sind keine Routing Updates gesendet
\item Optional (Router R1): Defaultroute an die anderen Router weiterverbreiten
\end{enumerate}


\begin{center}
\includegraphics[width=12cm]{img/13_rip.png}
\end{center}


\begin{center}
\includegraphics[width=12cm]{img/13_rip2.png}
\end{center}





\section{OSPF}
\subsection{Ziele von OSPF}
\begin{itemize}
\item Klassenlos: Adressräume mit variabel landen Subnetzmasken
\item Routing-Prozess muss robust sein
\item Routing-Prozess soll rasch konvergieren und rasch auf Änderungen reagieren
\item Protokoll soll skalieren
\item Bandbreite der Links soll berücksichtigt werden
\end{itemize}

\subsection{Vor- und Nachteile von LS Protokollen}

\paragraph{Vorteile}
\begin{itemize}
\item Jeder Router hat seine eigene Sicht des ganzen Netzes
\item Rasche Konvergenz
\item Updates nur bei Netzänderungen
\item Überwachung der Links mit Hello Paketen
\end{itemize}

\paragraph{Nachteile}
\begin{itemize}
\item Benötigt erheblich Memory für die LS-DB
\item Hohe Last auf der CPU, wenn der Dijkstra-Algorithmus gerechnet werden muss
\end{itemize}

\subsection{Konzepte von OSPF}
\begin{itemize}
\item OSPF als Schicht-3 Protokoll
\item Verschiedene Paket-Typen
\item Das Hello-Protokoll und den Aufbau von Nachbarschaften
\item Die endliche Zustandsmaschine beim Aufbau einer Nachbarschaft
\item Das Versenden von Link State Updates
\item Die Bestimmung der Router ID
\item Die Besonderheiten auf Multiaccess Netzen: Designated Router
\end{itemize}

\subsection{Funktionsweise}

\paragraph{Komponenten}
\emph{LS-Protokolle} tauschen die Routing-Informationen über definierte Paket-Typen aus. Für OSPF werden fünf Pakettypen unterschieden: \emph{1)} Hello Packet, \emph{2)} Database Description Packet, \emph{3)} Link-state Request Packet, \emph{4)} Link-state Update Packet, \emph{5)} Link-state Acknowledgement Packet.\\

\begin{center}
\begin{tabular}{ | m{0.5cm} | m{4cm}| m{5.5cm} | } 
Typ & Packet Name&Erklärung\\ 
\hline
1 & Hello Packet & Finden von Nachbaren und Aufbauen von Nachbarschaften; Wahl der Designated und des Backup Designated Routers\\
2 & Database Description (DBD) & Prüfen der Synchronisierung der Datenbank zwischen Routern; Router gibt bekannt von welchen Quellen die LS in der LSDB sind\\
3 & Link-State Request (LSR) & Fordert spezifische Link-State Einträge von Router zu Router\\
4 & Link-State Update (LSU) & Sendet spezifisch geforderte Link-State Einträge\\
5 & Link-State Acknowledgement (LSAck) & Bestätigung der anderen Packet Typen\\
\end{tabular}
\end{center}


Es werden drei Datenbanken geführt:
\begin{center}
\begin{tabular}{ | m{1.7cm} | m{2.5cm}| m{5.8cm} | } 
Datenbank & Tabelle&Erklärung\\ 
\hline
Adjacency DB & Nachbarschafts-Tabelle & Auflistung der Verbindungen zu OSPF Nachbarn\\
Liink-State DB & Topologie-Tabelle & Auflistung mit allen Routern und der daran angeschlossenen Links; Alle Router sollten die gleiche LSDB haben\\
Forwarding DB & Routing-Tabelle & Auflistung aller Zielnetze mit dem kürzesten Weg zu diesen Zielnetzen; Jeder Router hat eine eigene Routing-Tabelle\\
\end{tabular}
\end{center}

\paragraph{Ablauf LS-Protokoll}
\begin{enumerate}
\item \emph{Aufbau der Link State} -- Jeder Router sammelt die Infos über die direkt angeschlossenen Netze und deren Zustand und schreibt sie in die Link-State Datenbank (LSDB) respektive in die Topologie-Tabelle
\item \emph{Erstellen von Nachbarschaften} -- Jeder Router sendet allen Links, welche beim Routing mitmachen, Hello Pakete. Empfängt er \emph{Hello} Pakete von einem anderen Router, identifizieren sie sich gegenseitig, erstellen eine Nachbarschaft und tragen sich in ihre Nachbarschaftstabelle ein
\item \emph{Synchronisation der LSDB} -- Zwei Router tauschen die Informationen ihrer LSDB au. Nach der Synchronisation hat das LS-Protokoll synchronisiert
\item \emph{Berechnung der Routing-Tabelle} -- Jeder Router berechnet aus seiner Topologie-Tabelle den kürzesten Weg zu jedem Zielnetz und fügt die Ergebnisse der SPF-Berechnung in die Routing-Tabelle ein. Die Pfadkosten ergeben sich als Summe der Kosten jedes Links
\item \emph{Aktualisierung der Routing-Tabelle} -- Jeder Router, bei dem Änderungen der Link States auftreten, sendet ein \emph{Link-State Advertisement} an die eingetragenen Nachbarn. Die LSAs werden durch das ganze Internetwork geflutet. Jeder Router schreibt die Änderungen in seine LSDB und rechnet den Routing Algorithmus für die Zielnetze
\end{enumerate}

\paragraph{Funktionsweise} Das OSPF-Protokoll durchläuft folgende endliche Zustandsmaschine:
\begin{enumerate}
\item \emph{Down State} -- Router sendet Hello Packets; Souter hat noch keine Hello Packets erhalten
\item \emph{Init State} -- Router erhält Hello Packets auf Link; Router wertet die erhaltenen Informationen aus und initiiert eine Nachbarschaft
\item \emph{Two-Way State} -- Auf einem MA-Netz: Designated Router und Backup Designated Router auswählen
\item \emph{ExStart State} -- Aushandlung der Master-Slave Beziehung und der DBD Sequenznummer; Master initiiert DBD Austausch
\item \emph{Exchange State} -- Router tauschen DBD aus; In DBD-Paket beschreibt der Router, über welche Router er LS-Informationen in der LSDB hat;Falls einem Router LS fehlen, welche Neighbour hat, wechseln in Loading State; Falls alle LS hat, wechseln in Full State
\item \emph{Loading State} -- Mit LSRs verlangt der Router den Inhalt der LSDB des Nachbarn. Mittels LSU teilt ein Router den Inhalt der LSDB mit; Durchrechnen des SPF-Algorthmus
\item \emph{Full State} -- Das Routing-Protokoll auf den Routern läuft und hat konvergiert; Das Protokoll reagiert auf Änderungen im Netz
\end{enumerate}

\paragraph{Two-Way State} An einem broadcast multiaccess Netz können isch mehrere Router befinden -- Herausforderung für LS-Protokolle: $N * (N-1) / 2$ Nachbarschaften.\\

Es wird ein Designated und ein Backup Designated Router bestimmt. Alle anderen Router bauen eine Nachbarschaft zum DR und BDR auf aber keine zu den anderen. Sie senden ihre Updates an den DR und den BDR. Diese hören auf die Mcast-Adresse 224.0.0.6.\\

Der (B)DR wird Anhand von drei Kriterien gewählt: \emph{1)} Router mit der höchsten OSPF IF Priority wird DR, \emph{2)} Router mit der zweithöchsten OSPF IF Priority wird BDR; Falls gleich entscheidet die höchste Router-ID. Die Wahl beginnt sobald ein Netz auf einem Router für OSPF freigegeben wird. Sobald ein DR bestimmt ist, bleibt er dies, bis der Routing-Prozess neu gestartet wird.

\subsection{Metrik von OSPF und Parameter Bandwidth}
OSPF rechnet mit Kosten eines Links -- umgekehrt proportional der Datenrate des Links: \verb+cost  = reference-bandwidth / IF bandwidth+\\

Standardmässig wird die Referenz-Datenrate auf 100Mbps gesetzt. Die Referenz-Bandbreite kann konfiguriert werden (in Mbps -- Beispiel setzt auf 1 Gbps): \verb+R1(config router)#auto-cost reference-bandwidth 1000+\\
Die Einheit der Referenz-Datenrate ist Mbps.


\paragraph{Kosten von Serial Links}
Bei Ethernet-IF ist einem Router die Datenrate bekannt. Bei WAN-Links müssen (DTE-Ende) die Bandbreite (\emph{bandwidth}) konfiguriert werden -- \emph{Standardwert ist 1.544 Mbps}. Wenn der Link dem Standardwert entspricht, muss nicht angepasst werden.

Damit kein asymmetrisches Routing entsteht, müssen bei allen IF am gleichen Netz die gleiche Bandbreite eingestellt werden (vgl. Bild):
\begin{lstlisting}
// Router 1
R1(config)#interface serial0/0/1
R1(config-if)#bandwidth 64

// Router 2
R2(config)#interface serial0/0/1
R2(config-if)#bandwidth 1024

// Router 3
R3(config)#interface serial0/0/0
R3(config-if)#bandwidth 64
R3(config)#interface serial0/0/1
R3(config-if)#bandwidth 1024
\end{lstlisting}

\begin{center}
\includegraphics[width=12cm]{img/14_cost.png}
\end{center}


\subsection{Dijkstra-Algorithmus}
Berechnet über einer vermaschten Topologie einen Spannbaum als logische Topologie. Spannbaum gibt kürzesten Weg von einem Knoten aus zu allen anderen Knoten. Wenn diese Berechnung bekannt ist, kann der Weg zu den zugehörigen Netzen in der Routing-Tabelle eingetragen werden. Wird nur innerhalb einer Area berechnet.
\begin{lstlisting}
LS-Algorithmus fuer den Knoten u (Quelle):
1  Initialisierung:
2  N={u}
3  fuer alle Knoten v, die noch nicht in N enthalten sind:
4    wenn v ein Nachbar von u ist
5       dann D(v) = c(u,v)
6  sonst D(v) = inf
7  Wiederhole:
8    finde einen Knoten w, der noch nicht in N ist, so dass D(w)
9      minimal ist und fuege w zu N
10    Berechne D(v) neu fuer jeden Nachbarn v von w, der nicht in N ist:
11      D(v) = min(D(v), D(w) + c(w,v) )
12  /* Die neuen Kosten zu v sind entweder die alten Kosten zu v oder
13   die bekannten Kosten des kuerzesten Pfades zu w, zuzueglich
14   der Kosten von w zu v */
15 bis alle Knoten in N sind
\end{lstlisting}

\subsection{OSPFv2 und OSPF v3}
Wildcard Maske

\subsection{OSPF konfigurieren}
Router-ID
\begin{lstlisting}
R1(config)#router ospf 1
R1(config-router)#router-id 10.255.255.1
\end{lstlisting}

Passive Interface
\begin{lstlisting}
R2(config)#router ospf 1
R2(config-router)#passive-interface G0/0
\end{lstlisting}

Default Route
\begin{lstlisting}
// Definiere Default Route
R2(config)#ip route 0.0.0.0 0.0.0.0 Serial0/1/0
// Default Route im Netz bekanntgeben
R2(config)#router ospf 1
R2(config-router)#default-information originate
\end{lstlisting}

OSPFv3
\begin{lstlisting}
R1(config)#ipv6 unicast-routing
R1(config)#interface FastEthernet0/0
R1(config-if)#ipv6 address 2001:DB8:AB:1::1/64
R1(config-if)#ipv6 ospf 1 area 1
R1(config-if)#exit
R1(config)#ipv6 route ::/0 Serial0/0/0
R1(config)#ipv6 router ospf 1
R1(config-rtr)#auto-cost reference-bandwidth 10000
R1(config-rtr)#default-information originate
R1(config-rtr)#passive-interface g0/0
\end{lstlisting}





\section{Multiarea OSPF}
\subsection|{Wann ist es sinnvoll multi-area OSPF anzuwenden?}
\subsection{Router in Multi-Area Topologien aufteilen}
\subsection{LSA Typen}
Rechenaufwand auf eine Area beschränken

\subsection{MA OSPF für IPv4 und IPv6}
- konfigurueren
- Statusabfragen
\subsection{Routen zusammenfassen (für Bekanntgabe in andere Area)}

\end{document}












